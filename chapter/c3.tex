
\chapter{基于穿戴式IMU的运动监测算法设计}


\section{基于卷积神经网络的人类行为分类算法}


\section{步态分析算法}


\section{跌倒检测算法}


\section{本章小结}












% \subsection{数值算例与分析}
% 由于时域混合场积分方程是时域电场积分方程与时域磁场积分方程的线性组合,因此时域混合场积分方程时间步进算法的阻抗矩阵特征与时域电场积分方程时间步进算法的阻抗矩阵特征相同。

% \begin{algorithm}[H]
%     \KwData{this text}
%     \KwResult{how to write algorithm with \LaTeX2e}
%     initialization\;
%     \While{not at end of this document}{
%         read current\;
%         \eIf{understand}{
%             go to next section\;
%             current section becomes this one\;
%         }{
%             go back to the beginning of current section\;
%         }
%     }
%     \caption{How to wirte an algorithm.}
% \end{algorithm}

% 由于时域混合场积分方程是时域电场积分方程与时域磁场积分方程的线性组合,因此时域混合场积分方程时间步进算法的阻抗矩阵特征与时域电场积分方程时间步进算法的阻抗矩阵特征相同。

% \section{时域积分方程时间步进算法矩阵方程的求解}

% \section{本章小结}
% 本章首先研究了时域积分方程时间步进算法的阻抗元素精确计算技术,分别采用DUFFY变换法与卷积积分精度计算法计算时域阻抗元素,通过算例验证了计算方法的高精度。